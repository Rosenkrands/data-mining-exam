\documentclass[aspectratio=169,10pt]{beamer}
\usetheme[
%%% options passed to the outer theme
%    progressstyle=movCircCnt,   %either fixedCircCnt, movCircCnt, or corner
%    rotationcw,          % change the rotation direction from counter-clockwise to clockwise
%    shownavsym          % show the navigation symbols
  ]{AAUsimple}
  
% If you want to change the colors of the various elements in the theme, edit and uncomment the following lines
% Change the bar and sidebar colors:
%\setbeamercolor{AAUsimple}{fg=yellow!20,bg=yellow}
%\setbeamercolor{sidebar}{bg=red!20}
% Change the color of the structural elements:
%\setbeamercolor{structure}{fg=red}
% Change the frame title text color:
%\setbeamercolor{frametitle}{fg=blue}
% Change the normal text color background:
%\setbeamercolor{normal text}{fg=black,bg=gray!10}
% ... and you can of course change a lot more - see the beamer user manual.

\usepackage[utf8]{inputenc}
\usepackage[english]{babel}
\usepackage[T1]{fontenc}
% Or whatever. Note that the encoding and the font should match. If T1
% does not look nice, try deleting the line with the fontenc.
\usepackage{helvet}
\usepackage{pgffor}

% Pseudocode
\usepackage{xcolor}
\usepackage[linesnumbered,ruled,vlined]{algorithm2e}

% Highligthing
\newcommand{\hl}[1]{%
  \colorbox{yellow}{#1}
}

% Layout
\setlength{\parskip}{1em}

% colored hyperlinks
\newcommand{\chref}[2]{%
  \href{#1}{{\usebeamercolor[bg]{AAUsimple}#2}}%
}

\title{Data Mining Exam}

% \subtitle{}  % could also be a conference name

\date{\today}

\author{
  Kasper Rosenkrands
}

% - Give the names in the same order as they appear in the paper.
% - Use the \inst{?} command only if the authors have different
%   affiliation. See the beamer manual for an example

\institute[
%  {\includegraphics[scale=0.2]{aau_segl}}\\ %insert a company, department or university logo
  % Audio Analysis Lab, CREATE\\
  % Aalborg University\\
  % Denmark
] % optional - is placed in the bottom of the sidebar on every slide
{% is placed on the bottom of the title page
  %Department of Mathematical Sciences\\
  Aalborg University\\
  Denmark
  
  %there must be an empty line above this line - otherwise some unwanted space is added between the university and the country (I do not know why;( )
}

% specify a logo on the titlepage (you can specify additional logos an include them in 
% institute command below
\pgfdeclareimage[height=1.5cm]{titlepagelogo}{AAUgraphics/aau_logo_new} % placed on the title page
%\pgfdeclareimage[height=1.5cm]{titlepagelogo2}{AAUgraphics/aau_logo_new} % placed on the title page
\titlegraphic{% is placed on the bottom of the title page
  \pgfuseimage{titlepagelogo}
%  \hspace{1cm}\pgfuseimage{titlepagelogo2}
}

\begin{document}
% the titlepage
{\aauwavesbg%
\begin{frame}[plain,noframenumbering] % the plain option removes the header from the title page
  \titlepage
\end{frame}}
%%%%%%%%%%%%%%%%

% TOC
% \begin{frame}{Agenda}{}
% \tableofcontents
% \end{frame}
%%%%%%%%%%%%%%%%

\section{Clustering}

\subsection{Overview}
\begin{frame}{\secname}{\subsecname}
  \begin{itemize}
    \item What is clustering?
    \item K-Means optimization problem and algorithm
    \item Implementation of the K-Means algorithm and an example
    \item Hierarchical Clustering (briefly)
  \end{itemize}
\end{frame}

\subsection{Introduction}
\begin{frame}{\secname}{\subsecname}
  \textbf{Clustering} is a way to categorize data to impose structure.
  
  A use case is recommender systems (Amazon, Spotify, Netflix), where a user is recommended items that bought/listened to/watched by other users with similar interests.
\end{frame}

\subsection{K-Means Optimization Problem}
\begin{frame}{\secname}{\subsecname}
  % \begin{itemize}
  %   \item One example of a clustering algorithm is \textbf{K-Means}
  % \end{itemize}
  Given $D = (x_1, \ldots, x_n)$ where $x_i \in \mathbb{R}^p$, $K \in \mathbb{N}$ and let $C_1, \ldots, C_K$ denote different groups of the $x_i$'s.
  
  The K-Means algorithm tries to solve
  \begin{align}
    \min_{C_1, \ldots, C_K} \left\{\sum_{k=1}^K W(C_k)\right\}, \label{k-means-general-problem}
  \end{align}
  where $W(C_k)$ denotes the \textbf{within cluster variation}, in other words the dissimilarity of the group.
  \vspace{10pt}
  
  The most common dissimilarity measure is the is the squared Euclidean distance
  \begin{align}
    W(C_k) := \frac{1}{|C_k|} \sum_{i,i' \in C_k} \sum_{j = 1}^p (x_{i,j} - x_{i',j})^2.
  \end{align}
\end{frame}

\begin{frame}{\secname}{\subsecname}
  If we by $\bar{x}_{k,j} = \frac{1}{|C_k|}\sum_{i \in C_k} x_{i,j}$ denote the mean value of the $j$'th dimension in cluster $k$, it can be shown that
  \begin{align}
    \frac{1}{|C_k|} \sum_{i,i' \in C_k} \sum_{j = 1}^p (x_{i,j} - x_{i',j})^2 = 2 \sum_{i \in C_k} \sum_{j = 1}^p (x_{i,j} - \bar{x}_{k,j})^2.
  \end{align}
  If we further note that $\bar{x}_{k,j} = \min_{\mu_k} \left\{ \sum_{i \in C_k} \sum_{j = 1}^p (x_{i,j} - \mu_k)^2\right\}$ this implies that the optimization problem in \eqref{k-means-general-problem} can be rewritten as
  \begin{align}
    \min_{C_1, \ldots, C_k, \mu_1, \ldots, \mu_k} \left\{ \sum_{k = 1}^K \sum_{i \in C_k} \sum_{j = 1}^p (x_{i,j} - \mu_k)^2 \right\}.
  \end{align}
\end{frame}

\subsection{K-Means Algorithm}
\begin{frame}{\secname}{\subsecname}
  The K-Means algorithm is now able to exploit the new formulation of the optimization problem and iteratively solve for $\{C_1, \ldots, C_k\}$ and $\{\mu_1, \ldots, \mu_k\}$.

  This makes K-Means a greedy algorithm because, in each iteration it chooses optimal values for $\{C_1, \ldots, C_k\}$ and $\{\mu_1, \ldots, \mu_k\}$.

  Convergence of the algorithm is therefore ensured, however we cannot guarantee it will find the global optimum.
\end{frame}

\begin{frame}{\secname}{\subsecname}
  \begingroup
  \footnotesize
  \begin{algorithm}[H]
    \DontPrintSemicolon
    Assign each obsevation to a cluster randomly\;
    \ForEach{Cluster}{Compute the centroid\;}
    \ForEach{Observation}{
        Compute distance to all centroids\;
        Assign to the closest \;
    }
    \While{Centroids have not changed since last iteration}{
      \ForEach{Observation}{
        Compute distance to all centroids\;
        Assign to the closest \;
      }
      \ForEach{Cluster}{
        Compute the centroid\;
      }
    }
    \Return{Clusters}
    \caption{K-Means}
  \end{algorithm}
  \endgroup
\end{frame}

\subsection{Implementation of K-Means}
% \begin{frame}{\secname}{\subsecname}
% % my_kmeans <- function(data, K, animate = F, size = 3) {
% %   data$cluster <- as.factor(sample(c(1:K), size = nrow(data), replace = T))
% %   initial_cents <- calculate_cents(data, K)
  
% %   data_list <- list()
% %   data_list[[1]] <- data
% %   data_list[[2]] <- update_cluster(data, initial_cents)
  
% %   i = 1
% %   while (!identical(data_list[[i]]$cluster, data_list[[i + 1]]$cluster)) {
% %     cents <- calculate_cents(data_list[[i + 1]], K)
% %     data_list[[i + 2]] <- update_cluster(data_list[[i + 1]], cents)
% %     i = i + 1
% %   }
  
% %   cents <- calculate_cents(data_list[[i + 1]], K)
% % }
% \end{frame}

\subsection{An example of the K-Means algorithm}
\foreach \nn in{01,02,03,04,05,06,07,08,09,10,11,12,13}{
\begin{frame}{\secname}{\subsecname}
  \begin{figure}  
    \centering
    \includegraphics[width=.6\textheight]{scripts/kmeans-animation/animation_page_00\nn.png}
    \caption{Iteration \nn}
  \end{figure}
\end{frame}
}



\subsection{Hierarchical Clustering}
\begin{frame}{\secname}{\subsecname}
  \begin{center}
    \hl{Todo}
  \end{center}
  \begin{itemize}
    \item Introduction
    \item Type - Agglomerative vs Divise
    \item Pseudocode for algorithm, or just in words
    \item Visualization with dendogram
    \item Linkage types - (complete, single, average, centroid)
  \end{itemize}
\end{frame}

\section{Shrinkage}

\subsection{Overview}
\begin{frame}{\secname}{\subsecname}
  \begin{itemize}
    \item Variable Selection and Regularization (Because it is often cheaper to obtain multiple observations from a few samples than to obtain more samples, thus increasing the number of explanatory variables in a regression model, a linear regression model would be prone to increasing )
  \end{itemize}
\end{frame}

\begin{frame}{\secname}{Lasso}
\end{frame}

\begin{frame}{\secname}{Ridge Regression}
\end{frame}

\begin{frame}{\secname}{Elastic Net}
\end{frame}

\section{Classification}

\begin{frame}{\secname}{Linear Discriminant Analysis (LDA)}
\end{frame}

\begin{frame}{\secname}{Quadratic Discriminant Analysis (QDA)}
\end{frame}

\begin{frame}{\secname}{Naive Bayes}
\end{frame}

\section{Trees}

\begin{frame}{\secname}{Classification and Regression Trees (CART)}
\end{frame}

\begin{frame}{\secname}{Bagging}
\end{frame}

\begin{frame}{\secname}{Random Forest}
\end{frame}

\begin{frame}{\secname}{Boosting}
\end{frame}

\section{Support Vector Machines}

\begin{frame}{\secname}{}
\end{frame}

\section{Neural Networks}

\subsection{Perceptron}

\begin{frame}{\secname}{\subsecname}
    \textbf{Backpropagation}: For a given loss function $L$ we look for $\frac{\partial L}{\partial w_i}$.
    We start with initial values for the weights, which we shall denote $w_{old}$.
    Then we update the weights by $w_{new} = w_{old} - \eta \frac{\partial L}{\partial w}$.
    One iteration is called an \textbf{epoch}.
\end{frame}

{\aauwavesbg
\begin{frame}[plain,noframenumbering]
\end{frame}}
%%%%%%%%%%%%%%%%

\end{document}
