\documentclass[10pt]{beamer}
\usetheme[
%%% options passed to the outer theme
%    progressstyle=movCircCnt,   %either fixedCircCnt, movCircCnt, or corner
%    rotationcw,          % change the rotation direction from counter-clockwise to clockwise
%    shownavsym          % show the navigation symbols
  ]{AAUsimple}
  
% If you want to change the colors of the various elements in the theme, edit and uncomment the following lines
% Change the bar and sidebar colors:
%\setbeamercolor{AAUsimple}{fg=yellow!20,bg=yellow}
%\setbeamercolor{sidebar}{bg=red!20}
% Change the color of the structural elements:
%\setbeamercolor{structure}{fg=red}
% Change the frame title text color:
%\setbeamercolor{frametitle}{fg=blue}
% Change the normal text color background:
%\setbeamercolor{normal text}{fg=black,bg=gray!10}
% ... and you can of course change a lot more - see the beamer user manual.

\usepackage[utf8]{inputenc}
\usepackage[english]{babel}
\usepackage[T1]{fontenc}
% Or whatever. Note that the encoding and the font should match. If T1
% does not look nice, try deleting the line with the fontenc.
\usepackage{helvet}

% colored hyperlinks
\newcommand{\chref}[2]{%
  \href{#1}{{\usebeamercolor[bg]{AAUsimple}#2}}%
}

\title{Data Mining Exam}

% \subtitle{}  % could also be a conference name

\date{\today}

\author{
  Kasper Rosenkrands
}

% - Give the names in the same order as they appear in the paper.
% - Use the \inst{?} command only if the authors have different
%   affiliation. See the beamer manual for an example

\institute[
%  {\includegraphics[scale=0.2]{aau_segl}}\\ %insert a company, department or university logo
  % Audio Analysis Lab, CREATE\\
  % Aalborg University\\
  % Denmark
] % optional - is placed in the bottom of the sidebar on every slide
{% is placed on the bottom of the title page
  %Department of Mathematical Sciences\\
  Aalborg University\\
  Denmark
  
  %there must be an empty line above this line - otherwise some unwanted space is added between the university and the country (I do not know why;( )
}

% specify a logo on the titlepage (you can specify additional logos an include them in 
% institute command below
\pgfdeclareimage[height=1.5cm]{titlepagelogo}{AAUgraphics/aau_logo_new} % placed on the title page
%\pgfdeclareimage[height=1.5cm]{titlepagelogo2}{AAUgraphics/aau_logo_new} % placed on the title page
\titlegraphic{% is placed on the bottom of the title page
  \pgfuseimage{titlepagelogo}
%  \hspace{1cm}\pgfuseimage{titlepagelogo2}
}

\begin{document}
% the titlepage
{\aauwavesbg%
\begin{frame}[plain,noframenumbering] % the plain option removes the header from the title page
  \titlepage
\end{frame}}
%%%%%%%%%%%%%%%%

% TOC
\begin{frame}{Agenda}{}
\tableofcontents
\end{frame}
%%%%%%%%%%%%%%%%

\section{Clustering}

\begin{frame}{\secname}{Introduction}
  \begin{itemize}
    \item \textbf{Clustering} is a way to categorize data to impose structure.
    \item A use case is recommender systems (Amazon, Spotify, Netflix), where a user is recommended items that bought/listened to/watched by other users with similar interests.
  \end{itemize}
\end{frame}

\begin{frame}{\secname}{K-Means}
\end{frame}

\begin{frame}{\secname}{Hierarchical Clustering}
\end{frame}

\section{Shrinkage}

\begin{frame}{\secname}{Lasso}
\end{frame}

\begin{frame}{\secname}{Ridge Regression}
\end{frame}

\begin{frame}{\secname}{Elastic Net}
\end{frame}

\section{Classification}

\begin{frame}{\secname}{Linear Discriminant Analysis (LDA)}
\end{frame}

\begin{frame}{\secname}{Quadratic Discriminant Analysis (QDA)}
\end{frame}

\begin{frame}{\secname}{Naive Bayes}
\end{frame}

\section{Trees}

\begin{frame}{\secname}{Classification and Regression Trees (CART)}
\end{frame}

\begin{frame}{\secname}{Bagging}
\end{frame}

\begin{frame}{\secname}{Random Forest}
\end{frame}

\begin{frame}{\secname}{Boosting}
\end{frame}

\section{Support Vector Machines}

\begin{frame}{\secname}{}
\end{frame}

\section{Neural Networks}

\begin{frame}{\secname}{}
\end{frame}

{\aauwavesbg
\begin{frame}[plain,noframenumbering]
\end{frame}}
%%%%%%%%%%%%%%%%

\end{document}
